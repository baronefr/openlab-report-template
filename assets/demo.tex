\section{Introduction}

Hello world! If you are one of those exceptional minds that don't like to use Word, please be welcome and use this \LaTeX template for your project report! A preview of you face in this exact moment is presented in Figure~\ref{fig:example}.

\begin{figure}[h]
    \centering
    \includegraphics[width=0.25\textwidth]{assets/example.png}
    \caption{Example of a figure from Ref.~\cite{exampleCitation}}
    \label{fig:example}
\end{figure}

\lipsum[1]

\subsection{More}
More information is available for instance at \url{https://www.overleaf.com/learn/latex/Inserting_Images}.

\subsection{Even more}

\lipsum[2]

\subsubsection{More more, maybe?}

\lipsum[3-4]


\subsubsection{This is really the last one}

\lipsum[5]





\section{More but bigger}
Another section.
\ifprintCode
Custom environment \texttt{longlisting} used in Listing~\ref{code:example} enables page breaks within the code snippet.

\subsection{About listings}
More information on code snippets is available for instance at:
\begin{itemize}
    \item \url{https://www.overleaf.com/learn/latex/Code_Highlighting_with_minted},
    \item \href{https://repo.skni.umcs.pl/ctan/macros/latex/contrib/minted/minted.pdf}{minted package documentation.}
\end{itemize}


\begin{longlisting}
\begin{minted}[highlightlines={3}]{python}
import numpy as np
    
def incmatrix(genl1,genl2):
    m = len(genl1)
    n = len(genl2)
    M = None #to become the incidence matrix
    VT = np.zeros((n*m,1), int)  #dummy variable
    
    #compute the bitwise xor matrix
    M1 = bitxormatrix(genl1)
    M2 = np.triu(bitxormatrix(genl2),1) 

    for i in range(m-1):
        for j in range(i+1, m):
            [r,c] = np.where(M2 == M1[i,j])
            for k in range(len(r)):
                VT[(i)*n + r[k]] = 1;
                VT[(i)*n + c[k]] = 1;
                VT[(j)*n + r[k]] = 1;
                VT[(j)*n + c[k]] = 1;
                
                if M is None:
                    M = np.copy(VT)
                else:
                    M = np.concatenate((M, VT), 1)
                
                VT = np.zeros((n*m,1), int)
    
    return M
\end{minted}
\caption{Example code snippet~\cite{code}. \texttt{minted} package is required to compile it.}
\label{code:example}
\end{longlisting}
\fi




\subsection{Use LaTeX beautiful features}


Unlike Word, you can do very fancy stuff with \LaTeX. Please enjoy a vector sketch of a Paul trap...

\begin{center}
\scalebox{1.2}{
    \input{assets/demo-img}
}
\end{center}
... or this stunning Lindblad equation
\begin{equation}
    \dot{\rho} = \frac{i}{\hbar} [\rho,H] + \sum_j \gamma_j \left( L_j \, \rho \,  L_j^\dagger - \frac{1}{2} \Bigl\{ L_j^\dagger L_j, \, \rho \Bigr\} \right)
    \label{eq:Lindblad}
\end{equation}





% Use \section*{REFERENCES} to avoid REFERENCES being shown in the table of contents
\section*{REFERENCES}
\printbibliography[heading=none]

\noindent More information on bibliography is available for instance at \url{https://www.overleaf.com/learn/latex/Bibliography_management_with_biblatex}.

